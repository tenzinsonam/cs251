\documentclass[12pt, a4paper]{article}

\title{Movie Review: Gol Maal}
\author{}
\date{}

\begin{document}

\maketitle
Hrishikesh Mukherjee directs another comedy movie involving simple middle-class people and presents it through their everyday struggle. The entire picture is simple, authentic and fun. The concept of a young guy who would go to any extent to not lose his job is very interesting, and Mukherjee handles it with great success. Gol Maal is a comedy of errors and it's wonderfully developed and narrated. It's not a complex story, but it's not a silly one either, and the simplicity works in its favour. The film benefits from its genuinely lifelike sets and costumes, its script, and its proceedings, which grow from humorous to amusing to suspenseful to hilarious. It gets funnier and funnier as it goes along, as the conflict gets more complicated, the obstacles are doubled, and the irony grows. But the movie also has soul, and that's one of its high-points. The characters are colourful and memorable, but they also have depth which makes them easy to identify with. The relationships are also very impressively portrayed and it's nice to see how united families and relatives can be and how they care for each other and help each other in times of need.

\end{document}
